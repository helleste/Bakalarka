V rámci této práce jsem se snažil analyzovat problematiku multiplatformního vývoje mobilních aplikací ze všech možných úhlů. Na závěr se rozhodně nechci pouštět do vynášení obecných soudů, spíše bych rád shrnul poznatky, které jsem během studia dané problematiky načerpal a také se pokusil odhadnout, jaká budoucnost hybridní frameworky čeká.


Pro hybridní přístup je na trhu bezpochyby prostor, který se bude dle mého názoru nadále zvětšovat. Již v dnešní době máme na mobilním trhu tři nebo čtyři silné hráče, kteří jsou pro tvůrce aplikací zajímaví. V dalších letech se bude tento „long tail“ nadále zvětšovat zejména díky příchodu dalších menších hráčů. Mezi nimi si můžeme uvést například Firefox OS, jehož cílem je obsadit trh s low-endovými chytrými telefony na rozvojových trzích, dále Ubuntu Phone, který přenáší prostředí nejpopulárnější Linuxové distribuce na mobilní zařízení. Sledovat se vyplatí i projekt Tizen či znovuzrození webOS. 


Existuje tedy důvod domnívat se, že v přístích letech tu bude krom iOS, Androidu, Windows Phone a BlackBerry další silná skupina platforem s menším podílem, která však dohromady bude zabírat pro vývojáře zajímavou část trhu. Tento bobtnající „long tail“ tedy může být jedním z důvodů, proč bude hybridní přístup pro vývojáře stále atraktivnější. Je zajímavé si všimnout, že tito menší hráči jako Firefox OS nebo Ubuntu Phone přímo počítají s webovými/hybridními aplikacemi jakožto plnohodnotným přístupem k vývoji aplikací.


Na druhou misku vah je však třeba položit aktuální nezralost hybridních frameworků, která brání masovému přijetí tohoto přístupu mezi vývojáři. Hlavním problémem je především výkon. I na poměrně jednoduchých aplikacích lze poznat, že byly vyvinuty hybridní cestou. Především absence nativních UI elementů a nedostatečná rychlost JavaScriptových enginů jsou hlavními důvody, proč hybridní aplikace výkonově zaostávají. Musím na tomto místě opět připomenout výzkum společnosti Vision Mobile, kde se 29 \% vývojářů vyjádřilo, že právě problém s výkonem byl hlavním důvodem k opuštění hybridní cesty. Pro překonání tohoto problému je třeba mít hluboké znalosti CSS a JavaScriptových frameworků, což ovšem může výrazně zvýšit náklady na tvorbu aplikace.


Bude zajímavé sledovat, jak se s narůstajícím podílem hybridních aplikací vyrovnají tvůrci současných lídrů na trhu mobilních platforem. Jejich zájmem totiž rozhodně není, aby se aplikace tvořily jednotně pro všechny platformy. Pokud vývojář napíše aplikaci exkluzivně pro jednu platformu, dává jí to zajímavou výhodu před konkurencí. Dobrým příkladem je například populární Instagram, který měl po dlouhou dobu aplikaci pouze pro systém iOS. 


Tvůrci těchto platforem přitom mají nad hybridními aplikacemi velkou moc. Nemusí jim povolit přístup na svá oficiální tržiště či jim mohou komplikovat život pomocí úprav webového runtime, který je pro většinu hybridních frameworků životně důležitý. Prvním náznakem takového omezování je přístup společnosti Apple, která poskytuje vývojářům webový runtime s pomalejším JavaScriptovým enginem, než jaký sama využívá ve svém vlastním mobilním prohlížeči. Pravděpodobně tedy můžeme očekávat další podobné překážky, které budou hybridním frameworkům kladeny do cesty, v rámci snahy o zachování důležité konkurenční výhody.


Velký potenciál pro hybridní přístup vidím v segmentu firemních aplikací. Takzvaný BYOD (Bring Your Own Device) přístup začíná nabývat na síle i u nás a otvírá tím hybridním aplikacím dveře dokořán. U firemních aplikací, které jsou využívány primárně zaměstnanci k přístupu k datům z informačních systémů, totiž příliš nezáleží na vzhledu ani na špičkovém výkonu mobilní aplikace. Důležitější je multiplatformní pokrytí a nízké náklady, tedy kritéria, v nichž hybridní přístup vyniká. Není náhodou, že Demokratická strana ve Spojených státech využívala hybridní aplikace ke koordinaci dobrovolníků pracujících na kampani Baracka Obamy. Důvod je zřejmý. Každý dobrovolník si mohl takovou aplikaci nainstalovat na svůj chytrý telefon nehledě na to, jaký operační systém používá. A stranu to přišlo na minimální náklady.


Budoucnost hybridních frameworků tedy skýtá mnoho příležitostí, které mohou posunout tento přístup ještě více do centra zájmu vývojářů mobilních aplikací. K tomu, aby se hybridní aplikace mohly směle poměřovat s těmi nativnímí, musí však urazit ještě velmi dlouhou cestu. V příštích letech se jako uživatelé budeme jistě nadále setkávat především s nativními aplikacemi. To platí zejména pro herní tituly a další výkonově náročnější aplikace. Dynamický vývoj HTML5, nárust „long tailu“ na trhu s mobilními platformami a rozšířování BYOD přístupu ve firmách však dává hybridnímu přístupu příležitost vychytat mnohé nedostatky a stát se v budoucnu plnohodnotnou cestou, kterou budou vývojáři vyvíjet mobilní aplikace.